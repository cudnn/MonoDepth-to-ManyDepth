\documentclass[conference]{IEEEtran}
\IEEEoverridecommandlockouts
% The preceding line is only needed to identify funding in the first footnote. If that is unneeded, please comment it out.
\usepackage{cite}
\usepackage{amsmath,amssymb,amsfonts}
\usepackage{algorithmic}
\usepackage{graphicx}
\usepackage{textcomp}
\usepackage{xcolor}
\usepackage{verbatim}
\usepackage{algorithm2e}
\def\BibTeX{{\rm B\kern-.05em{\sc i\kern-.025em b}\kern-.08em
    T\kern-.1667em\lower.7ex\hbox{E}\kern-.125emX}}
\begin{document}

\title{}

\author{\IEEEauthorblockN{Seri Lee}
\IEEEauthorblockA{\textit{Computer Science and Engineering} \\
\textit{Seoul National University}\\
Seoul, Republic of Korea \\
sally20921@snu.ac.kr}
}

\maketitle

\begin{abstract}
\end{abstract}

\begin{IEEEkeywords}
\end{IEEEkeywords}

\section{Introduction}
\cite{b1}. 

\subsection{Geometry and Acquisition of a Single Image}
\subsection*{Simple Camera System: the Pin-hole Model}
The simplest form of real camera comprises a pinhole and an imaging scene (or plane). Because the pinhole lies between the imaging screen and the observed 3D world scene, any ray of light that is emitted or reflected from a surface patch in the scene is contrained to travel through the pinhole before reaching the imaging screen. Therefore, there is correspondence between each 2D area on the imaing screen and the area in the 3D world, as observed through the pinhole from the imaging screen. A mathematical model of the simple pin-hole camera is illustrated in Figure 3.6. Notice that the imaing screen is now in front of the pinhole. This formulation simplified the concept of projection to that of magnification. In order to understand how points in the real world are related mathematically to points on the imaging screen two coordinate systems are of particular interest: 1. The external coordinate system, which is independent of placement and parameters of the camera. 2. The camera coordinate sytsem. The two coordinate systems are related by a translation, expressed by matrix $T$, and rotation, represented by matrix $R$.

The point $O_c$ called a central or a focal point, together with the axes $X_c, Y_c$ and $Z_c$ determine the coordinate system of the camera. An important part of the camera model is the image plane. We can observe in Figure 3.6 that this plane has been tesellated into rectangular elements, i.e. tiled, and that within an electronic camera implementation these tiles will form discrete photosening locations that smaple any image projected onto the plane. Each tile is called a pixel, i.e. picture element, and is indexed by a pair of coordinates expressed by natural numbers .Figure 3.6 depicts the plane with a discrete grid of pixels. The projection of the point $O_c$ on the plane in the direction of $Z_c$ determines the principal point of local coordinate $(o_x, o_y)$. 

Point $p$ is an image of point $P$ under the projection with a center in point $O_c$ on to the plane. Coordinates of the points $p$ and $P$ in the camera coordinate system are denoted as follows: $P = [X Y Z]$, $p = [x y z]$. 

\subsection*{Extrinsic Parameters}
The mathematical description of a given scene depends on the chosen coordinate system. With respect to the chosen coordinate system and based solely on placement of the image plane we determine an exact placement of the camera. Thereafter, it is often practical to select just the camera coordinate system as a reference. The situation becomes yet more complicated, however, if we have more than one camera since the exact (relative) position of each camera must be determined.

A change from the camera coordinate system to the external world coordinate system can be accomplisehd providing a translation $T$ and a rotation $R$. The translation vector $T$ describes a change int he position of the coordinate centers $O_c$ and $O_w$. The rotation, in turn, changes the corresponding axes of each system. This change is described by the orthogonal matrix $R$ of dimensions 3x3.

For a given poitn $P$, its coordinates related to the camera and external coordinates related to the external world are connected by the following formula: $P_c = R(P_w - T)$, where $P_c$ expresses placement of a point $P$ in the camera coordinate system, $P_w$ is its placement in the external coordinate system, $R$ stands for the rotation matrix and $T$ is the translation matrix between origins of those two coordinate systems. 

Summarizing, we say that the extrinsic parameters of the perspective camera are all the necessary geometric parameters that allow a change from the camera coordinate system to the external coordinate system and vice versa. Thus, the extrinsic parameters of a camera are just introduced matrices $R$ and $T$. 

\subsection*{Intrinsic Parameters}
The intrinsic camera parameters can be summarized as follows. The parameters of the projective transformation itself: For the pin-hole camera model, this is given by the focal length $f$. The parameters that map the camera coordinate sysem into the image coordinate system: Assuming that the origin of the image constitutes a point $o = (o_x, o_y)$ and that the physical dimensions of pixels on a camera plane in the two directions are constant and given by $h_x$ and $h_y$, a relation between image coordinates $x_u$ and $y_u$ and camera coordinates $x$ and $y$ can be stated as follows: $x = (x_u - o_x)h_x$, $y = (y_u - o_y)h_y$.



\bibliography{refs}
\bibliographystyle{plain}

\end{document}

